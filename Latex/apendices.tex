\begin{appendices}
\chapter{Indicadores de bolsa}\label{apendice}
En esta secci\'on se entra en profundidad en los indicadores usados en la realizaci\'on de las condiciones situadas en los nodos de los \'arboles de decisi\'on. La intenci\'on de este anexo es, por tanto, ilustrar como se calcula cada indicador para caracterizarlo, pero no se entrar\'a en sus distintas interpretaciones cl\'asicas.\\


\noindent\textbf{SMA \textit{(Simple Moving Average)}}\\

Es un indicador cl\'asico y, como su propio nombre indica, es una media aritm\'etica de los \'ultimos valores.\\

Tiene un \'unico par\'ametro que corresponde con el n\'umero de instantes a incluir en la media. As\'i pues, si notamos a $P_i$ como el precio de la acci\'on $i$ instantes atr\'as, el c\'alculo del \textit{SMA} ser\'ia el siguiente:

\[SMA(period) = \frac{\sum\limits_{i=1}^{period}P_i}{period}\]

\vspace{0.5cm}
\noindent\textbf{EMA \textit{(Exponential Moving Average)}}\\

La media m\'ovil exponencial es un indicador parecido al \textit{SMA}. Pero, en esta ocasi\'on, los precios de los instantes anteriores no van a tener el mismo peso en la media, es decir, es una media ponderada. Con este indicador, los valores m\'as cercanos en el tiempo son m\'as importantes.\\

La f\'ormula general del \textit{EMA} viene dada por

\[EMA(period) = K_{period} * P_0 + (1 - K_{period}) * EMA_{[1]}\]

donde $P_0$ es el valor actual de la acci\'on, $EMA_{[1]}$ es el valor del $EMA$ en el instante anterior y $K_{period} \in (0,1)$ es un valor que depende del periodo escogido. Habitualmente, se toma $K_{period} = \frac{2}{period + 1}$. 

\vspace{0.5cm}
\noindent\textbf{MACD \textit{(Moving Average Convergence Divergence)}}\\

El nombre de este indicador es algo confuso ya que realmente no es un media, sino una diferencia de medias. En concreto, es la diferencia de dos \textit{EMA} de periodo distinto.\\

En consecuencia, \textit{MACD} tiene dos par\'ametros, los periodos de las dos medias exponenciales. El periodo peque\~no debe ser extrictamente menor que el periodo grande.

\[MACD(period_h, period_l) = EMA(period_l) - EMA(period_h)\] 

\vspace{0.5cm}
\noindent\textbf{ATR \textit{(Average True Range)}}\\

Este indicador intenta medir la volatilidad\footnote{La volatilidad es un concepto burs\'atil que hace referencia a la rapidez con la que cambia un determinado valor en un periodo fijo de tiempo. Una volatilidad alta suele ser s\'intoma de inseguridad en los inversores o de un cambio de tendencia.} del precio de la acci\'on. Para ello toma un par\'ametro, el periodo, que ser\'a el que marcar\'a la longitud del intervalo. Es necesario tener, para cada instante del periodo, el precio m\'as alto ($H$), el precio m\'as bajo ($L$) y el precio de cierre ($C$). Se define, entonces, el \textit{TR (True Range)} como 

\[TR = max\{ H-L, |H-C|, |L-C|\}\]

Una vez calculado este valor para todos los instantes del periodo, el \textit{ATR} viene dado por

\[ATR(period) = \frac{1}{period}\sum\limits_{i=0}^{period-1}TR_{i}\]

donde $TR_{i}$ es el \textit{True Range} $i$ instantes atr\'as.

\vspace{0.5cm}
\noindent\textbf{TRANGE \textit{(True Range)}}\\

Es un caso espec\'ifico del indicador anterior cuando el periodo es 1, no obstante, resulta bastante provechoso incluirlo de nuevo. El \textit{TRANGE} es un estad\'istico que indica la diferencia entre el precio m\'aximo y el precio m\'inimo de una acci\'on en un instante. Por tanto, es un indicador que marca el nivel de volatilidad.\\


\vspace{0.5cm}
\noindent\textbf{ROC \textit{(Price Rate Of Change)}}\\

El indicador \textit{ROC} nos da un porcentaje de cambio del precio de la acci\'on respecto de un instantes anterior. Tiene un solo par\'ametro, llamado periodo, que indica la distancia del instante anterior a tomar respecto del instante actual.\\

Para calcular este indicador podemos usar la f\'ormula siguiente, donde $C$ es el valor de cierre del instante actual y $C_{period}$ es el valor de cierre del instante $period$ veces atr\'as.

\[ROC(period) = \frac{C - C_{period}}{C_{period}}\]

\vspace{0.5cm}
\noindent\textbf{Momento}\\

Este indicador es el m\'as simple de todos, simboliza la diferencia del precio de la acci\'on entre el momento actual y un instante anterior. Tiene como par\'ametro la distancia entre el instante actual y el anterior.\\

Su c\'alculo, que es bastante sencillo, puede realizarse como

\[Momento(period) = C - C_{period}\] 


\vspace{0.5cm}
\noindent\textbf{Hill}\\

Este es un indicador propuesto en este proyecto. Es similar al Momento, pero con una diferencia, se divide por el periodo. Por tanto, el indicador representa la variaci\'on del precio por instante.\\

Al analizarlo, se deduce que este indicador es equivalente al Momento, ya que se pueden construir las mismas condiciones a partir de \'el. No obstante, aporta una ventaja, el pivote situado en la condici\'on tiene una magnitud relativa, es decir, cambiar el par\'ametro del indicador Hill sin cambiar el pivote mantiene la informaci\'on. Esta caracter\'ista, por contrario, no se tiene con el Momento. En nuestro caso, este indicador podr\'ia estar mejor preparado para las mutaciones de par\'ametros.\\ 

Como cabe esperar, su c\'alculo viene dado por

\[Hill(period) = \frac{Momento(period)}{period} = \frac{C - C_{period}}{period}\] 


\noindent\textbf{RSI \textit{(Relative Strengh Index)}}\\

Este indicador se usa, habitualmente, para detectar cu\'ando hay un periodo de sobreventa o sobrecompra.\footnote{Un periodo de sobrecompra es un periodo en el que se han comprado muchas acciones, provocando una subida del precio. La sobreventa es el concepto an\'alogo. Sendos periodos suelen continuarse con un rebote o correcci\'on, es decir, se efect\'ua el periodo contrario con menos intensidad.} A pesar de que tiene un s\'olo par\'ametro, el periodo, su c\'alculo es algo m\'as complejo que el del resto. Se definen dos funciones an\'alogas:

\[AverageGain(period) = \frac{\sum\limits_{\substack{i=1 \\ C_i - O_i > 0}}^{period}\frac{C_i-O_i}{O_i}}{period}\]

\[AverageLoss(period) = \frac{\sum\limits_{\substack{i=1 \\ C_i - O_i < 0}}^{period}\frac{O_i-C_i}{O_i}}{period}\]

Como se observa, son la media de las ganacias en los d\'ias de ganacia y la media de las p\'erdidas en los d\'ias de p\'erdida, respectivamente. Una vez calculados los dos valores, el indicador RSI se calcula como

\[RSI(period) = 100 - \frac{100}{1+\frac{AverageGain(period)}{AverageLoss(period)}}\]

\vspace{0.5cm}
\noindent\textbf{OBV \textit{(On-Balance Volume)}}\\

Este es un indicador basado en el volumen de transacciones, es decir, la cantidad de acciones que se compran o venden en un instante. No tiene ningun par\'ametro y, por tanto, la mutaci\'on de par\'ametros no altera nada.\\

La f\'ormula viene dada por

\[   
OBV = OBV_{1} +
\begin{cases}
\text{volume} &\quad\text{si}\quad C < C_{1}\\
\text{0} &\quad\text{si}\quad   C = C_{1}\\
\text{-volume} &\quad\text{si}\quad  C < C_{1}\\ 
\end{cases}
\]

donde $OBV_{1}$ es el valor de \textit{OBV} en el instante anterior y $C$ y $C_1$ son los precios de cierre actual y anterior, respectivamente.\\
\newpage

\noindent\textbf{AD \textit{(Accumulation/Distribution)}}\\

El indicador \textit{AD} es un indicador basado tanto en volumen como en el precio. Intenta resumir como se distribuyen las compras de acciones en funcion del volumen. Este indicador, al igual que el anterior, no recibe par\'ametros.\\

Su c\'alculo viene dado por

\[AD = AD_{1} + \frac{(C - L) - (H - C)}{H - L}*Volume \]

donde, como anteriormente, la sub\'indice 1 denota el instante anterior.

\vspace{0.5cm}
\noindent\textbf{BBANDS\_HIGH y BBANDS\_LOW \textit{(Bollinger Bands)}}\\

Este indicador, compuesto a su vez por dos indicadores, propone construir dos valores, uno superior y otro inferior, que flanqueen al precio de la acci\'on en cada instante. La distancia de estos valores al precio depende del \textit{SMA} y la desviaci\'on t\'ipica de los precios t\'ipicos. A continuaci\'on, se muestra el c\'alculo de los dos valores, pero primero son necesarias algunas definiciones.\\

Los precios t\'ipicos son una correci\'on del precio calculada como la media del precio de cierre, el m\'aximo y el m\'inimo, es decir, $TP = (C + H + L) / 3$. Ahora si, las \textit{Bollinger Bands} se definen como

\[BBAND\_HIGH = SMA_{TP}(period) + m*\sigma_{TP}(period)\] 
\[BBAND\_LOW = SMA_{TP}(period) - m*\sigma_{TP}(period)\] 

donde $SMA_{TP}(period)$ es la media m\'ovil de los precios t\'ipicos de longitud $period$ y $\sigma_{TP}(period)$ es la desviaci\'on t\'ipica de los precios t\'ipicos.\\

Sin embargo, no se han usado directamente estos valores, los indicadores usados en este proyecto miden la distancia entre el precio y el valor de la \textit{Bollinger Band} y, adem\'as, est\'an normalizados por la apertura de la misma. Esto es

\[D\_HIGH = \frac{H - BBAND\_HIGH}{ BBAND\_HIGH - BBAND\_LOW}\] 
\[D\_LOW =  \frac{L - BBAND\_HIGH}{ BBAND\_HIGH - BBAND\_LOW}\]

donde $H$ y $L$ son los valores m\'aximo y m\'inimo del precio del instante.

\end{appendices}
